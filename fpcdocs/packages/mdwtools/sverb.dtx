% \begin{meta-comment}
%
% $Id: sverb.dtx,v 1.1 2000/07/13 09:10:21 michael Exp $
%
% Verbatim typesetting done properly (ahem)
%
% (c) 1996 Mark Wooding
%
%----- Revision history -----------------------------------------------------
%
% $Log: sverb.dtx,v $
% Revision 1.1  2000/07/13 09:10:21  michael
% + Initial import
%
% Revision 1.1  1998/09/21 10:19:01  michael
% Initial implementation
%
% Revision 1.3  1996/11/19 21:01:18  mdw
% Entered into RCS
%
%
% \end{meta-comment}
%
% \begin{meta-comment} <general public licence>
%%
%% sverb package -- handling of verbatim text
%% Copyright (c) 1996 Mark Wooding
%%
%% This program is free software; you can redistribute it and/or modify
%% it under the terms of the GNU General Public License as published by
%% the Free Software Foundation; either version 2 of the License, or
%% (at your option) any later version.
%%
%% This program is distributed in the hope that it will be useful,
%% but WITHOUT ANY WARRANTY; without even the implied warranty of
%% MERCHANTABILITY or FITNESS FOR A PARTICULAR PURPOSE.  See the
%% GNU General Public License for more details.
%%
%% You should have received a copy of the GNU General Public License
%% along with this program; if not, write to the Free Software
%% Foundation, Inc., 675 Mass Ave, Cambridge, MA 02139, USA.
%%
% \end{meta-comment}
%
% \begin{meta-comment} <Package preamble>
%<+package>\NeedsTeXFormat{LaTeX2e}
%<+package>\ProvidesPackage{sverb}
%<+package>                [1996/05/08 1.3 Verbatim typesetting]
% \end{meta-comment}
%
% \CheckSum{651}
%% \CharacterTable
%%  {Upper-case    \A\B\C\D\E\F\G\H\I\J\K\L\M\N\O\P\Q\R\S\T\U\V\W\X\Y\Z
%%   Lower-case    \a\b\c\d\e\f\g\h\i\j\k\l\m\n\o\p\q\r\s\t\u\v\w\x\y\z
%%   Digits        \0\1\2\3\4\5\6\7\8\9
%%   Exclamation   \!     Double quote  \"     Hash (number) \#
%%   Dollar        \$     Percent       \%     Ampersand     \&
%%   Acute accent  \'     Left paren    \(     Right paren   \)
%%   Asterisk      \*     Plus          \+     Comma         \,
%%   Minus         \-     Point         \.     Solidus       \/
%%   Colon         \:     Semicolon     \;     Less than     \<
%%   Equals        \=     Greater than  \>     Question mark \?
%%   Commercial at \@     Left bracket  \[     Backslash     \\
%%   Right bracket \]     Circumflex    \^     Underscore    \_
%%   Grave accent  \`     Left brace    \{     Vertical bar  \|
%%   Right brace   \}     Tilde         \~}
%%
%
% \begin{meta-comment}
%
%<*driver>
%
% $Id: mdwtools.ins,v 1.1 2000/07/13 09:10:21 michael Exp $
%
% Installer for the mdwtools packages
%
% (c) 1996 Mark Wooding
%

%----- Revision history -----------------------------------------------------
%
% $Log: mdwtools.ins,v $
% Revision 1.1  2000/07/13 09:10:21  michael
% + Initial import
%
% Revision 1.1  1998/09/21 10:19:01  michael
% Initial implementation
%
% Revision 1.3  1996/11/19 20:57:26  mdw
% Entered into RCS
%

% --- Licence note ---
%
% mdwtools installer
% Copyright (c) 1996 Mark Wooding
%
% This program is free software; you can redistribute it and/or modify
% it under the terms of the GNU General Public License as published by
% the Free Software Foundation; either version 2 of the License, or
% (at your option) any later version.
%
% This program is distributed in the hope that it will be useful,
% but WITHOUT ANY WARRANTY; without even the implied warranty of
% MERCHANTABILITY or FITNESS FOR A PARTICULAR PURPOSE.  See the
% GNU General Public License for more details.
%
% You should have received a copy of the GNU General Public License
% along with this program; if not, write to the Free Software
% Foundation, Inc., 675 Mass Ave, Cambridge, MA 02139, USA.

% --- Sort out how to do all this ---

\def\batchfile{mdwtools.ins}
\input docstrip
\keepsilent

\preamble

IMPORTANT NOTICE
\endpreamble

{ % --- This is a group so that docstrip \reads it all in one go ---

  \ifx\generate\mdwxxnotdef
    \gdef\mdwgen#1{#1}
    \gdef\mdwf#1#2{\generateFile{#1}{n}{#2}}
    \gdef\needed#1{}
  \else
    \global\let\mdwgen\generate
    \global\def\mdwf{\file}
    \global\askforoverwritefalse
  \fi

}

\mdwgen{\mdwf {at.sty}		{\from {at.dtx}	      {package}}
	\mdwf {mdwlist.sty}	{\from {mdwlist.dtx}  {package}}
	\mdwf {mdwtab.sty}	{\from {mdwtab.dtx}   {mdwtab}
				 \needed{syntax.dtx}
				 \from {footnote.dtx} {macro}
				 \from {doafter.dtx}  {macro}}
	\mdwf {syntax.sty}	{\from {syntax.dtx}   {package}
				 \from {doafter.dtx}  {macro}}
	\mdwf {mathenv.sty}	{\from {mdwtab.dtx}   {mathenv}}
	\mdwf {mdwmath.sty}	{\from {mdwmath.dtx}  {package}}
	\mdwf {sverb.sty}	{\from {sverb.dtx}    {package}}
	\mdwf {footnote.sty}	{\from {footnote.dtx} {package}}
	\mdwf {doafter.sty}	{\from {doafter.dtx}  {package,latex2e}}
	\mdwf {doafter.tex}	{\from {doafter.dtx}  {package,plain}}
	\mdwf {cmtt.sty}	{\from {cmtt.dtx}     {sty}}
	\mdwf {mTTenc.def}	{\from {cmtt.dtx}     {def}}
	\mdwf {mTTcmtt.fd}	{\from {cmtt.dtx}     {fd}}
}

\Msg{Done!}

\describespackage{sverb}
\mdwdoc
%</driver>
%
% \end{meta-comment}
%
% \section{User guide}
%
% The \package{sverb} package provides some useful commands and environments
% for doing things with verbatim text.  I prefer this code to the standard
% \package{verbatim} package (by Rainer Sch\"opf et al.)\ although I'm
% biased.
%
% The package was written to fulfil a particular purpose: I wanted to be able
% to typeset ARM assembler code, 77~columns wide, on A5~paper, with the
% fields separated by \textit{tab} characters.  It's grown up fairly
% organically from that, and I've tidied it when I've seen the code get too
% ugly.
%
% The current features are:
%
% \begin{itemize}
%
% \item A `listing' environment which typesets verbatim text nicely.
%
% \item A command to read verbatim text from an external file.
%
% \item Support for arbitrary-sized chunks of text without overflowing \TeX's
%       memory.
%
% \item Support for \textit{tab} characters in the verbatim text.
%
% \item An environment for typesetting demonstrations of \LaTeX\ markup.
%
% \item It all works correctly with the \package{doc} system for documenting
%       \LaTeX\ packages.
%
% \item A fairly hairy but quite powerful programmer interface to the yukky
%       bits of the package.
%
% \end{itemize}
%
% The interface is described in its own section, so that more timid readers
% can avoid it.  That said, some of the stuff in this section gets rather
% technical.
%
% Note that this package doesn't even try to do anything with short bits of
% verbatim text (as handled by the |\verb:...:| command).  I have a separate
% package (\package{syntax}) which does all sorts of horrible things along
% those lines.
%
% \subsection{The \env{listing} environment}
%
% \DescribeEnv{listing}
% The main method for typesetting verbatim text is the \env{listing}
% environment.  This works pretty much the same as the standard
% \env{verbatim} environment, with some exceptions, which are described
% below.
%
% So that you know exactly what you're getting, here are the rules by which
% \package{sverb} decides what the verbatim text actually is:
%
% \begin{itemize}
%
% \item If there's any text, other than spaces, on the same line as the
%       `|\begin{listing}|', then the contents of the environment begins
%       immediately after the closing brace (with all leading spaces
%       preserved).  Otherwise, the text begins on the following line.
%
% \item If there is any text, other than spaces, before the
%       `|\end{listing}|', but on the same line, this is considered to be the
%       last line of the text; otherwise the text is presumed to have ended
%       at the end of the previous line.
%
% \item Any text following the |\end{listing}| on the same line is thrown
%       away.  There are good reasons for this, but they're technical.
%       Essentially there's nothing I can do about it.
%
% \end{itemize}
%
% \begin{figure}
% \begin{demo}[w]{The \env{listing} environment}
%\dots in the following code:
%
%\begin{listing}
%init            MOV     R0,#200         ;Version 2.00 please
%                LDR     R1,=&4B534154   ;Magic number (`TASK')
%                ADR     R2,appName      ;Find application name
%                SWI     Wimp_Initialise ;Register as a WIMP task
%\end{listing}
%
%The next step is to \dots
% \end{demo}
% \end{figure}
%
% Tab characters are supported within the environment: tab stops are set
% every eighth column, although this can be modified.
%
% \subsubsection{Configuring the \env{listing} environment}
%
% The text size used in the \env{listing} environment is set by the
% |\listingsize| command.  By default, this is set to |\small|, although you
% can redefine it in the document preamble, or it can be set in the document
% class.
%
% The amount by which the listing text is indented is controlled by the
% |\listingindent| length parameter.  This is a fixed length, whose default
% value is 1\,em.
%
% \subsubsection{Choosing a different end-text}
%
% \DescribeEnv{listing*}
% The \env{listing} environment is terminated by the exact character sequence
% `|\end{listing}|'.  This isn't too much of a problem, unless you want to
% include this string in the text.  This is achieved by the \env{listing$*$}
% environment, which allows you to specify the end-text to find as an
% argument.
%
% For example:
%
% \begin{demo}{The \env{listing$*$} environment}
%Type a listing as follows:
%
%\begin{listing*}{<end-listing*>}
%\begin{listing}
%This is a listing.  Yes.
%\end{listing}
%<end-listing*>
%\end{demo}
%
% Don't include `special' characters in your chosen end-text unless you know
% what you're doing.
%
% \subsection{Writing text to a file}
%
% \DescribeEnv{verbwrite}
% You can write verbatim text to a file using the \env{verbwrite}
% environment.  The syntax is fairly straightforward:
%
% \begin{quote}
% \syntax{"\\begin{verbwrite}{"<file-name>"}" \dots "\\end{verbwrite}"}
% \end{quote}
%
% The text of the environment is written to the named file.  The rules about
% where the text actually starts and ends are the same as for the
% \env{listing} environment.
%
% There is also a $*$-variant, like \env{listing$*$}, which allows you to
% choose the end-text.  The end-text is the first argument, the filename
% comes second.
%
% There is a restriction on the characters you can write to the file: they
% must all be considered `printable' by \TeX; otherwise they will be read
% back in as `\syntax{"^^"<chars>}' which isn't too good.  Unfortunately,
% this includes tab characters, so you can't write them.\footnote{^^A
%   Well, not without doing serious surgery on \TeX\ itself, anyway. }
%
% \iffalse [Example time...  Ho hum.  There is evilness here.] \fi
%\begin{verbwrite*}{<end-write>}{wrdemo1.tmp}
%\begin{verbwrite}{wrdemo.tmp}
%This is some text written to
%a file near the beginning of
%the file.
%\end{verbwrite}
%<end-write>
%
% For example: \verbinput{wrdemo1.tmp}
%
% \input{wrdemo1.tmp} \iffalse [Now build the file ;-) ] \fi
%
% \subsection{The \cmd\verbinput\ command}
%
% \DescribeMacro{\verbinput}
% You can input a pre-prepared text file exactly as it is in the input using
% the |\verbinput| command.  The filename is given as an argument.  For
% example:
%
% \begin{demo}{The \cmd\verbinput\ command}
%\verbinput{wrdemo.tmp}
% \end{demo}
%
% \subsection{The \env{demo} environment}
%
% Package authors need to document their packages, and it's common to want
% to display examples showing the original text and the output side-by-side
% (or, when space doesn't permit this, one above the other).  Both the
% \LaTeX\ book and \textit{The \LaTeX\ Companion} contain such examples.
%
% The \env{demo} environment allows such displays to be created easily.  The
% syntax of the environment is as follows:
%
% \begin{quote}
% \syntax{"\\begin{demo}["<shape>"]{"<title>"}" \dots "\\end{demo}"}
% \end{quote}
%
% The optional \synt{shape} argument can be either `|w|' (wide), or `|n|'
% (narrow).  A `wide' shape places the input and output one above the other,
% while the `narrow' shape puts them side-by-side.  The default shape is
% `narrow'.  An attractive border is drawn around the display to finish it
% off nicely.
%
% An example:
%
%\begin{demo*}{<end-demo>}[w]{The \env{demo} environment}
%\begin{demo}{From the \textit{\TeX book}}
%\[ \sum_{p\;\rm prime}
%   f(p) = \int_{t>1}
%      f(t)\,{\rm d}\pi(t) \]
%\end{demo}
%<end-demo>
%
% \DescribeEnv{demo*}
% As with the other environments created by this package, there's a
% $*$-variant which takes the end-text as an argument.
%
%
% \section{Programmer interface}
%
% This section describes the publicly available routines provided by the
% \package{sverb} package.  Routines not described here are libable to be
% changed or even removed without warning, so don't use them.
%
% \subsection{Environment hooks}
%
% Each of the environments created here works in the same way.  For each
% environment \env{foo}, there's a main command responsible for doing the
% work, called |\sv@foo|.  This is given all the arguments of the normal
% environment, and two more:
%
% \begin{itemize}
%
% \item The `end-text' to search for, which marks the end of the environment.
%
% \item Some actions to perform after the text has been read and processed.
%       This allows the calling macro to do some extra actions, like closing
%       boxes, etc.
%
% \end{itemize}
%
% All the environments do is call the main command with appropriate
% arguments.
%
% \subsection{Reading the verbatim text}
%
% \DescribeMacro{\sv@read}
% The main scanning routine is |\sv@read|.  It is called with three
% arguments:
%
% \begin{itemize}
%
% \item The end-text marking the end of the environment.
%
% \item The name of a macro (which must be a single token) which is called
%       with a line of text as its single argument.  This is given each
%       line of text which is read from the environment in turn.
%
% \item A macro, or other sort of action, which is to be done when the text
%       has been read and processed.
%
% \end{itemize}
%
% The macro |\sv@read| assumes that the caller has already made some
% provision for removing the category codes of the following text, by either
% calling |\@verbatim| or using the construction
% \begin{listing}
%\let\do=\@makeother
%\dospecials
% \end{listing}
%
% \DescribeMacro{\sv@safespc}
% Note that any space characters you read using |\sv@read| will be catcoded
% as |\active|.  Normally this is OK because |\obeyspaces| (or
% |\@vobeyspaces|) will be in effect.  If you're doing something more exotic,
% like writing text to a file or building a command string, you can call
% |\sv@safespc| which defines the active-space character to be a normal
% whitespace-space when expanded.
%
% \implementation
%
% \section{Implementation}
%
% This section defines several macros and environments which allow verbatim
% typing, with a high degree of configurability.  OK, so this sort of
% thing's been done so often before that it isn't true, but I don't really
% care.
%
%    \begin{macrocode}
%<*package>
%    \end{macrocode}
%
% \subsection{Simple things}
%
% To help us build funny macros which involve strange and different category
% codes, I'll write some simple macros which I can use while building my
% complicated and clever ones.
%
% \begin{macro}{\@cspecials}
%
% This macro is used to assist the definition of some of the environments.
% It makes `|\|', `|{|' and `|}|' into `other' characters, and replaces them
% with `\verb"|"', `|<|' and `|>|' respectively.  Note that `|[|' and `|]|'
% aren't used, because they make defining commands which take optional
% arguments awkward.  Note that we open a group here.  This should be closed
% using \verb"|endgroup" at the end of the special section.
%
%    \begin{macrocode}
\def\@cspecials{%
  \begingroup%
  \catcode`|0%
  \catcode`<1%
  \catcode`>2%
  \catcode`\{12%
  \catcode`\}12%
  \catcode`\\12%
}
%    \end{macrocode}
% \end{macro}
%
% \begin{macro}{\sv@startlisting}
%
% This macro sets everything up nicely for a \env{listing}-type verbatim
% environment.
%
%    \begin{macrocode}
\def\sv@startlisting{%
  \def\par{\@@par\penalty\interlinepenalty}%
  \@@par%
  \leftskip\@totalleftmargin%
  \obeylines%
  \@noligs%
  \let\do\@makeother\dospecials%
  \verbatim@font%
  \frenchspacing%
  \@vobeyspaces%
  \settabwidth%
  \catcode9\active%
  \lccode`\~9\lowercase{\let~\sv@vtab}%
  \lccode`\~13\lowercase{\let~\vinput@cr}%
  \interlinepenalty500%
}
%    \end{macrocode}
%
% \end{macro}
%
% \subsection{Tab character handling}
%
% One of the things we want to do here is handle tab characters properly.
% (Here, `properly' means `moving to the next column which is a multiple of
% eight', the way these things were always meant to.)
%
% \begin{macro}{\settabwidth}
%
% The tabs used by our tabbed verbatim environments are set up by this
% routine.  It sets the tab width parameter |\svtab| to 8 times the width
% of a |\tt| space.  If you really want, you can redefine this macro.
%
%    \begin{macrocode}
\newdimen\svtab
\def\settabwidth{\setbox\z@\hbox{\texttt{\space}}\svtab8\wd\z@}
%    \end{macrocode}
%
% \end{macro}
%
% \begin{macro}{\sv@vtab}
%
% Here we handle tabs inside verbatim environments.  We expect each line to
% be typeset as a box, using something like
%
% \begin{listing}
%\setbox0\hbox{#1}
%\leavevmode
%\box0
%\par
% \end{listing}
%
% The idea is that you make tab active, and set it to this macro.  We stop
% the current box, stretch it to the right width, and start another one
% straight after, so nobody know the difference.  The code here is straight
% from Appendix~D of \textit{The \TeX book}.
%
%    \begin{macrocode}
\def\sv@vtab{%
  \hfill\egroup%
  \@tempdima\wd\z@%
  \divide\@tempdima\svtab%
  \multiply\@tempdima\svtab%
  \advance\@tempdima\svtab%
  \wd\z@\@tempdima%
  \leavevmode\box\z@%
  \setbox\z@\hbox\bgroup%
}
%    \end{macrocode}
%
% \end{macro}
%
% \begin{macro}{\verbinput}
%
% We allow input from a file, by the |\verbinput| command.  We display the
% text pretty much the same as the \env{listing} environment below.
%
% We set tab and return active, and get them to do appropriate things.  This
% isn't actually all that hard.
%
%    \begin{macrocode}
\def\verbinput#1{%
  \begin{listinglist}%
  \listingsize%
  \sv@startlisting%
  \setbox\z@\hbox\bgroup%
  \input{#1}%
  \sv@stripspc%
  \egroup%
  \ifdim\wd\z@=\z@%
    \ifhmode\par\fi%
  \else%
    \leavevmode\box\z@\par%
  \fi%
  \end{listinglist}%
}
%    \end{macrocode}
%
% \end{macro}
%
% \begin{macro}{\vinput@cr}
%
% This macro handles return characters while inputting text in |\verbinput|.
% We just output our current box, and start another.
%
%    \begin{macrocode}
\def\vinput@cr{%
  \egroup%
  \leavevmode\box\z@%
  \par%
  \setbox\z@\hbox\bgroup%
}
%    \end{macrocode}
%
% \end{macro}
%
% \subsection{Reading verbatim text}
%
% The traditional way of reading verbatim text is to use a delimited
% argument, as described in the \textit{\TeX book}.  This works well-ish if
% the text isn't very long.  A better solution would be to pick out the text
% line-by-line and process it like that.  So this is what we do.
%
% \begin{macro}{\matcher}
%
% For long verbatim environments, we need to be able to find the end text.
% This is rather tricky.  The solution here is rather horrible.  The
% environment picks out each line of the text at a time, as an argument, and
% tests to see if it contains the text we're after.  We do the test in a
% particularly yukky way: we add the actual target text to the end of the
% line, and inspect the text following the match to see if the match is at
% the end.
%
% The |\matcher| macro creates a `matcher' which will test strings to see if
% they contain something interesting.
%
% To create a matcher, say
% \syntax{"\\matcher{"<cmd-name>"}{"<target>"}{"<process-cmd>"}"}.  The
% command \synt{cmd-name} accepts a line of text as an argument and calls
% the \synt{process-cmd} with the text of the line before the match, or the
% whole lot.  It also sets |\@ifmatched| appropriately.
%
% (Having spent ages coming up with this cruft myself, I found some very
% similar, but slightly better, code in Appendix~D.  So I've changed mine to
% match Donald's.  Anyway, credit where it's due: cheers Don.)
%
%    \begin{macrocode}
\newif\if@matched
\def\matcher#1#2#3{%
  \expandafter\def\csname\string#1$match\endcsname##1#2##2##3\end{%
    \ifx##2\relax%
      \@matchedfalse%
    \else%
      \@matchedtrue%
    \fi%
    #3{##1}%
  }%
  \expandafter\def\expandafter#1\expandafter##\expandafter1\expandafter{%
    \csname\string#1$match\endcsname##1#2\relax\end%
  }%
}
%    \end{macrocode}
%
% \end{macro}
%
% \begin{macro}{\sv@stripspc}
%
% This macro strips any trailing glue in the current horizontal list.  This
% is fairly simple, actually: we just loop while glue is the last item.  It's
% slightly complicated by penalties which \TeX\ puts into the list between
% the glue items, but we just remove them too.
%
%    \begin{macrocode}
\def\sv@stripspc{%
  \unpenalty%
  \ifdim\lastskip=\z@\else%
    \unskip\expandafter\sv@stripspc%
  \fi%
}
%    \end{macrocode}
%
% \end{macro}
%
% \begin{macro}{\sv@percent}
%
% This macro strips a single leading percent character if there is one, and
% if the \env{doc} package is loaded.  We store the possibly stripped text in
% |\@tempa|.
%
%    \begin{macrocode}
\begingroup
\catcode`\%=12
\gdef\sv@percent#1#2\relax
    {\ifx\check@percent\@@undefined
                     \ifx#1\relax\def\@tempa{}\else
                         \def\@tempa{#1#2}\fi\else
                     \ifx#1\relax\def\@tempa{}\else
                         \ifx#1%\def\@tempa{#2}\else
                             \def\@tempa{#1#2}\fi\fi\fi}
\endgroup
%    \end{macrocode}
%
% \end{macro}
%
% \begin{macro}{\@isspaces}
%
% We want to avoid writing the first and last lines of the environment to the
% file if there's nothing in them.  To do this, we need to know whether a
% piece of text contains only space characters.  This macro does this, in a
% rather nasty way.  See the other macros below for details of how this
% works.
%
% We define |\sv@safespc| at the same time: this makes space active and
% expand to a space character which is not active.  Neat, huh?
%
%    \begin{macrocode}
\lccode`\~32
\lccode`\!32
\lowercase{%
  \def\@isspaces#1{%
    \ifx#1\relax%
      \def\@tempb{\@tempswafalse}%
    \else\ifx#1~%
      \let\@tempb\@isspaces%
    \else%
      \def\@tempb##1\relax{}%
    \fi\fi%
    \@tempb%
  }
  \def\sv@safespc{%
    \catcode32\active%
    \def~{ }%
  }
}
%    \end{macrocode}
%
% \end{macro}
%
% \begin{macro}{\sv@read}
%
% This macro does the main job of reading a chunk of verbatim text.  You call
% it like this:
%
% \begin{quote}
% \syntax{"\\sv@read{"<end-text>"}{"<process-line-proc>"}{"<end-proc>"}"}
% \end{quote}
%
% The \synt{end-text} is the text to find at the end of the `environment': we
% stop when we find it.
%
% The \synt{process-line-proc} is a macro which is passed as an argument each
% line which we read from the text.
%
% The \synt{end-proc} is a macro to call once we've finished reading all of
% the text.  This can tidy up an environment or close a file or whatever.
%
% We read the text by picking out newlines using a delimited macro.  We have
% to be a little clever, because newlines are active in verbatim text.
%
% We will also strip `|%|' signs off the beginning if the \package{doc}
% package is here (\package{doc} tries to play with \LaTeX's verbatim stuff,
% and doesn't understand the way we do things).
%
%    \begin{macrocode}
\def\sv@read#1#2#3{%
%    \end{macrocode}
%
% This code does all sorts of evil things, so I'll start by opening a group.
%
%    \begin{macrocode}
  \begingroup%
%    \end{macrocode}
%
% So that I can spot the end-text, I'll create a matcher macro.
%
%    \begin{macrocode}
  \matcher\@match{#1}\sv@read@ii%
%    \end{macrocode}
%
% So that I can identify line ends, I'll make them active.  I'll also make
% spaces active so that they can expand to whatever they ought to expand
% to (spaces in files, or funny \verb*" " characters or whatever.
%
%    \begin{macrocode}
  \catcode13\active%
  \catcode32\active%
%    \end{macrocode}
%
% I'll use the |\if@tempswa| flag to tell me whether I ought to output the
% current line.  This is a little messy, so I'll describe it later.  I'll
% initialise it to false because this is the correct thing to do.
%
%    \begin{macrocode}
  \@tempswafalse%
%    \end{macrocode}
%
% Most of the job is done by two submacros.  I'll define them in terms of
% my current arguments (to save lots of token munging).  The first just
% extracts the next line (which ends at the next newline character) and
% tries to match it.
%
%    \begin{macrocode}
  \lccode`\~13\lowercase{%
    \def\sv@read@i##1~{\@match{##1}}%
  }%
%    \end{macrocode}
%
% The results of the match get passed here, along with the text of the
% line up to the matched text.
%
%    \begin{macrocode}
  \def\sv@read@ii##1{%
%    \end{macrocode}
%
% The first job to do is to maybe strip off percent signs from the beginning,
% to keep \package{doc} happy.
%
%    \begin{macrocode}
    \sv@percent##1\relax\relax%
%    \end{macrocode}
%
% Now I need to decide whether I ought to output this line.  The method goes
% like this: if this is the first line (|\if@tempswa| is false) or the last
% (|\if@matched| is true), \emph{and} the text consists only of spaces, then
% I'll ignore it.
%
% The first thing to do is to notice the last line -- if |\if@matched| is
% true, then I'll make |\if@tempswa| false to make the first-line and
% last-line cases work the same way.
%
%    \begin{macrocode}
    \if@matched\@tempswafalse\fi%
%    \end{macrocode}
%
% Now if this is the first or last line, I'll examine it for spaces.  This
% is done in a separate macro.  It will set |\if@tempswa| false if the
% text contains only spaces.
%
%    \begin{macrocode}
    \if@tempswa\else\@tempswatrue\expandafter\@isspaces\@tempa\relax\fi%
%    \end{macrocode}
%
% Now, if |\if@tempswa| is still true, perform the \<process-line-proc> on
% the line of text.  I'll provide a group, so that it doesn't upset me
% too much.
%
%    \begin{macrocode}
    \if@tempswa%
      \begingroup%
      \expandafter#2\expandafter{\@tempa}%
      \endgroup%
    \fi%
%    \end{macrocode}
%
% The next line won't be the first one, so I'll set the flag true in
% readiness.
%
%    \begin{macrocode}
    \@tempswatrue%
%    \end{macrocode}
%
% Now, if that wasn't the last line, go round again; otherwise end the group
% I started ages ago, and do the user's \<end-proc>.
%
%    \begin{macrocode}
    \if@matched\def\@tempa{\endgroup#3}\else\let\@tempa\sv@read@i\fi%
    \@tempa%
  }%
%    \end{macrocode}
%
% Now to start the thing up.  I'll read the first line.
%
%    \begin{macrocode}
  \sv@read@i%
}
%    \end{macrocode}
%
% \end{macro}
%
% \begin{macro}{\sv@readenv}
%
% This macro works out an appropriate end-text for the current environment.
% If you say \syntax{"\\sv@readenv{"<macro-name>"}"}, it will expand do
% \begin{listinglist} \listingsize \synshorts
% <macro-name>"{\\"$_{12}$"end{"$_{12}$<current-env-name>"}"$_{12}$"}"^^A
%		"{\\end{"<current-env-name>"}}"
% \end{listinglist}
% Easy, no?
%
% This is all done with mirrors.  No, err\dots\ it's done with
% |\expandafter|.
%
%    \begin{macrocode}
\begingroup
\lccode`\<=`\{
\lccode`\>=`\}
\lccode`\|=`\\
\lowercase{\endgroup
\def\sv@readenv#1{%
  \expandafter\expandafter\expandafter%
    #1\expandafter\sv@readenv@i\@currenvir\@@%
}
\def\sv@readenv@i#1\@@{{|end<#1>}{\end{#1}}}
}
%    \end{macrocode}
%
% \end{macro}
%
% \begin{macro}{\sv@verbline}
%
% This macro typesets a line in a verbatim way, so you can construct a real
% verbatim environment from it.  It's a bit tricky in the way that it catches
% the last line.  Don't worry about this: it's easy really.  Note the
% |\relax| after the |\par| -- this is because \package{doc} tries to do
% clever things with |\par| to strip `|%|' signs out.
%
%    \begin{macrocode}
\def\sv@verbline#1{%
  \setbox\z@\hbox{#1\sv@stripspc}%
  \ifdim\wd\z@=\z@%
    \if@matched\ifhmode\par\relax\fi\else\leavevmode\par\relax\fi%
  \else%
    \leavevmode\box\z@\par\relax%
  \fi%
}
%    \end{macrocode}
%
% \end{macro}
%
% \subsection{Listing environments}
%
% The \env{listing} environment is our equivalent of the standard
% \env{verbatim} environment.  We do some slightly cleverer things, though,
% to make sure (for example) that even text which contains |\end{listing}|
% can be typeset.
%
% \begin{macro}{\listinglist}
% \begin{environment}{listinglist}
%
% This defines the layout for the \env{listing} environment.  It starts a
% list with the appropriate shape.  It's also made into an environment, so
% that the end-paragraph-environment bits work correctly.
%
% The |\listingindent| length parameter sets up the indentation of the
% listings.  If there's a |\parindent| setting, I'll line listings up with
% that; otherwise I'll just choose something which looks right.
%
%    \begin{macrocode}
\newdimen\listingindent
\AtBeginDocument{%
  \ifdim\parindent=\z@\listingindent1em\else\listingindent\parindent\fi%
}
%    \end{macrocode}
%
% Now to define a size hook for the environment.  This is fairly simple
% stuff.
%
%    \begin{macrocode}
\ifx\listingsize\@@undefined
  \let\listingsize\small
\fi
%    \end{macrocode}
%
% Now to define the environment itself.  Suppress the indentation if we're
% first thing on a new list item, so that the listing lines up with
% everything else.
%
%    \begin{macrocode}
\def\listinglist{%
  \list{}{%
    \if@inlabel%
      \leftmargin\z@%
    \else%
      \leftmargin\listingindent%
    \fi%
    \rightmargin\z@%
    \labelwidth\z@%
    \labelsep\z@%
    \itemindent\z@%
    \listparindent\z@%
    \let\makelabel\relax%
    \parsep\z@skip%
  }%
  \parfillskip\@flushglue%
  \item\relax%
}
\let\endlistinglist\endlist
%    \end{macrocode}
%
% \end{environment}
% \end{macro}
%
% \begin{environment}{listing}
%
% The \env{listing} environment is the only real verbatim-like environment we
% create will all this kit, although it does the job very nicely.
%
% The environment indents its contents slightly, unlike \env{verbatim}, and
% uses a smaller typeface in an attempt to fit 77-column text on an A5~page.
% There is also a $*$-variant, which allows you to specify the terminating
% text.  This enables you to include absolutely any text in the environment,
% including |\end{listing}|.
%
% First, we must define the |\listing| command.
%
%    \begin{macrocode}
\def\listing{%
  \listinglist%
  \listingsize%
  \sv@readenv\sv@listing%
}
%    \end{macrocode}
%
% Now we define the |\@listing| command, which does most of the work.  We
% base the \env{listing} environment on a \env{list}.
%
%    \begin{macrocode}
\def\sv@listing#1#2{%
  \sv@startlisting%
  \sv@read{#1}\sv@verbline{\endlistinglist#2}%
}
%    \end{macrocode}
%
% Now we define the starred version.  The command name needs to include the
% `|*|' character, so we must use |\csname|.  There's some hacking here to
% allow us to read the name using the appropriate catcodes for otherwise
% normal characters: \LaTeX\ activates some characters and makes them typeset
% themselves to suppress some ligaturing.
%
%    \begin{macrocode}
\expandafter\def\csname listing*\endcsname{%
  \listinglist%
  \listingsize%
  \begingroup%
  \@noligs%
  \def\@tempa##1{\endgroup\sv@listing{##1}{\end{listing*}}}%
  \@tempa%
}
%    \end{macrocode}
%
% \end{environment}
%
% \begin{environment}{ignore}
%
% The \env{ignore} environment entirely ignores its contents.  Anything at
% all may be put into the environment: it is discarded utterly.
%
% We define some macros for defining ignoring environments, because this can
% be useful for version control, possibly.
%
%    \begin{macrocode}
\def\sv@ignore#1#2{%
  \@bsphack%
  \let\do\@makeother\dospecials%
  \sv@read{#1}\@gobble{\@esphack#2}%
}
\def\ignore{\sv@readenv\sv@ignore}
\def\ignoreenv#1{%
  \expandafter\let\csname #1\endcsname\ignore%
}
\def\unignoreenv#1{%
  \expandafter\def\csname #1\endcsname{\endgroup}%
  \expandafter\def\csname end#1\endcsname%
      {\begingroup\def\@currenvir{#1}}%
}
%    \end{macrocode}
%
% \end{environment}
%
% \subsection{The \env{verbwrite} environment}
%
% The \env{verbwrite} environment allows text to be written to a file in a
% verbatim way.  Note that tab characters don't work, because \TeX\ refuses
% to be nice.
%
% \begin{macro}{\sv@write}
%
% As seems to be traditional now, we first define a general hookable macro
% which allows a caller to specify the end-text and what to do afterwards.
%
%    \begin{macrocode}
\newwrite\sv@writefile
\def\sv@write#1#2{%
  \begingroup%
  \@bsphack%
  \let\do\@makeother\dospecials%
  \sv@safespc%
  \sv@read{#1}\sv@writeline{\sv@endwrite#2}%
}
\def\sv@writeline#1{%
  \immediate\write\sv@writefile{#1}%
}
\def\sv@endwrite{%
  \@esphack%
  \endgroup%
}
%    \end{macrocode}
%
% \end{macro}
%
% \begin{environment}{verbwrite}
%
% Now we can define the actual environment.  We define a $*$-variant which
% allows the user to specify the end-text, just to make sure.
%
%    \begin{macrocode}
\def\verbwrite#1{%
  \immediate\openout\sv@writefile#1\relax%
  \sv@readenv\sv@write%
}
\def\endverbwrite{\immediate\closeout\sv@writefile}
\expandafter\def\csname verbwrite*\endcsname#1#2{%
  \immediate\openout\sv@writefile#2\relax%
  \sv@write{#1}{\immediate\closeout\sv@writefile\end{verbwrite*}}%
}
%    \end{macrocode}
%
% \end{environment}
%
% \subsection{The \env{demo} environment}
%
% By way of tying all of this together, I present an environment for
% displaying demonstrations of \LaTeX\ markup.  We read the contents of the
% environment, write it to a temporary file, and read it back twice,
% typesetting it the first time and displaying it verbatim the second time.
%
% \begin{macro}{\sv@demoname}
%
% This macro expands to the filename to use for the temporary data.  To
% allow the package documentation to demonstrate the \env{demo} environment
% itself, we need to keep a nesting count.  This avoids too much hackery,
% which unfortunately appears to plague all of my \TeX\ code.
%
%    \begin{macrocode}
\newcount\sv@nestcount
\def\sv@demoname{demo\number\sv@nestcount.tmp}
%    \end{macrocode}
%
% \end{macro}
%
% \begin{macro}{\sv@demo}
%
% As for listing, we do all the business through a private macro.  This is
% good because it means we can leave the main macro readable.  The argument
% is the end-text to spot.
%
%    \begin{macrocode}
\def\sv@demo#1#2{%
  \@ifnextchar[{\sv@demo@i{#1}{#2}}{\sv@demo@i{#1}{#2}[n]}%
}
\def\sv@demo@i#1#2[#3]#4{%
  \advance\sv@nestcount by\@ne%
  \immediate\openout\sv@writefile\sv@demoname\relax%
  \sv@write{#1}{%
    \immediate\closeout\sv@writefile%
    \sv@dodemo{#2}{#3}{#4}%
  }%
}
%    \end{macrocode}
%
% \end{macro}
%
% \begin{environment}{demo}
%
% This is the real environment.  We provide \env{demo$*$} too, to allow the
% user to choose the end-text.
%
%    \begin{macrocode}
\def\demo{\sv@readenv\sv@demo}
\expandafter\def\csname demo*\endcsname#1{\sv@demo{#1}{\end{demo*}}}
%    \end{macrocode}
%
% \end{environment}
%
% \begin{macro}{\sv@dodemo}
%
% First, let's define some common bits of code in the stuff below.  The
% minipages used to typeset the material has some clever stuff to avoid
% strange spacing in the output.
%
%    \begin{macrocode}
\def\sv@demosmp{%
  \begin{minipage}[t]{\@tempdima}%
  \vskip8\p@%
  \hrule\@height\z@%
  \raggedright%
  \vbox\bgroup%
}
\def\sv@demoemp{%
    \par\unpenalty\unskip%
  \egroup%
  \vskip8\p@%
  \hrule\@height\z@%
  \end{minipage}%
}
%    \end{macrocode}
%
% This is the macro which actually typesets the demonstration.
%
%    \begin{macrocode}
\def\sv@dodemo#1#2#3{%
%    \end{macrocode}
%
% Now work out some values.  We set |\hsize| to the line width leaving 2\,em
% of space on either side.  The size of the minipages is calculated depending
% on the shape of the demonstration.  This is all fairly simple.
%
%    \begin{macrocode}
  \begingroup%
  \@tempdima\linewidth%
  \advance\@tempdima-2em%
  \hsize\@tempdima%
  \if#2w%
    \advance\@tempdima-2em%
  \else%
    \advance\@tempdima-3em%
    \divide\@tempdima2%
  \fi%
%    \end{macrocode}
%
% Now we open a big vertical box, and put in a header to mark off the
% demonstration.
%
%    \begin{macrocode}
  \par%
  \setbox\z@\hbox{\strut\enspace#3\enspace\strut}%
  \@tempdimb.5\dp\z@%
  \advance\@tempdimb-.5\ht\z@%
  \ht\z@\@tempdimb\dp\z@\@tempdimb%
  \noindent\hskip1em\vtop{%
    \hb@xt@\hsize{%
      \hrulefill%
      \raise\@tempdimb\box\z@%
      \hrulefill%
    }%
    \nointerlineskip%
    \hb@xt@\hsize{\vrule\@height5\p@\hfil\vrule\@height5\p@}%
    \nointerlineskip%
%    \end{macrocode}
%
% Now we insert the output text in the first minipage.  I'll force `|%|'
% to be a comment character, in case something like \package{doc} has had its
% wicked way.
%
%    \begin{macrocode}
    \vskip-\parskip%
    \noindent\hbox{}\hskip1em%
    \sv@demosmp%
    \catcode`\%14\relax%
    \input{\sv@demoname}%
    \sv@demoemp%
%    \end{macrocode}
%
% Insert some kind of separation between the two.  In `wide' format, we start
% a new line, and put a ruleoff between the two.  In `narrow' format, we just
% leave some space.
%
%    \begin{macrocode}
    \if#2w%
      \vskip8\p@\hrule\vskip8\p@%
      \noindent\hbox{}%
    \fi%
    \hskip1em%
%    \end{macrocode}
%
% Now we put the verbatim copy of the text in the other minipage.
%
%    \begin{macrocode}
    \sv@demosmp%
    \listingindent\z@%
    \verbinput\sv@demoname%
    \sv@demoemp%
    \par%
    \nointerlineskip%
    \hb@xt@\hsize{\vrule\@height5\p@\hfil\vrule\@height5\p@}%
    \hrule%
  }%
  \endgroup%
  \par%
  \vskip\baselineskip%
  #1%
}
%    \end{macrocode}
%
% \end{macro}
%
% That's all there is.  Have fun.
%
%    \begin{macrocode}
%</package>
%    \end{macrocode}
%
% \hfill Mark Wooding, \today
%
% \Finale
%
\endinput
